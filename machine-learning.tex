% Created 2019-09-16 一 09:29
\documentclass[11pt]{article}
\usepackage[utf8]{inputenc}
\usepackage[T1]{fontenc}
\usepackage{fixltx2e}
\usepackage{graphicx}
\usepackage{longtable}
\usepackage{float}
\usepackage{wrapfig}
\usepackage{rotating}
\usepackage[normalem]{ulem}
\usepackage{amsmath}
\usepackage{textcomp}
\usepackage{marvosym}
\usepackage{wasysym}
\usepackage{amssymb}
\usepackage{hyperref}
\tolerance=1000
\author{king}
\date{\today}
\title{机器学习网上参考材料}
\hypersetup{
  pdfkeywords={},
  pdfsubject={},
  pdfcreator={Emacs 25.2.2 (Org mode 8.2.10)}}
\begin{document}

\maketitle
\tableofcontents


\section{算法与数据结构}
\label{sec-1}

\subsection{动态规划}
\label{sec-1-1}
\begin{itemize}
\item 动态规划经典题目汇总 \url{https://blog.csdn.net/dengwei4321/article/details/14137295}
\item 面试中的动态规划 \url{https://blog.csdn.net/qq_15288217/article/details/80693352}
\item 面试经典动态规划问题 \url{https://blog.csdn.net/u013007900/article/details/56298701}
\end{itemize}

\subsection{图搜索}
\label{sec-1-2}
\begin{itemize}
\item 图/树——宽度优先搜索(BFS)\url{https://blog.csdn.net/Echo1214_Xie/article/details/81672803}
\end{itemize}

\section{大数据技术}
\label{sec-2}

\subsection{Flink技术关键}
\label{sec-2-1}
\begin{itemize}
\item 第一次有人把Apache Flink说的这么明白! \url{https://www.jianshu.com/p/c5949402134c}
\item Flink案例代码,面试题 \url{https://blog.csdn.net/qq_31866793/article/details/90755173}
\end{itemize}

\subsection{推荐类算法}
\label{sec-2-2}

\subsubsection{协同过滤}
\label{sec-2-2-1}
\begin{itemize}
\item 基于物品的协同过滤算法(ItemCF) \url{https://www.cnblogs.com/dtstack/p/10271551.html}
\item 推荐系统之基于用户的协同过滤算法(UserCF) \url{https://blog.csdn.net/sinat_35866463/article/details/79428898}
\item 皮尔森相关性的相似度 \url{https://blog.csdn.net/ifnoelse/article/details/7765984}
\end{itemize}

\section{机器学习基础}
\label{sec-3}

\subsection{综合机器学习}
\label{sec-3-1}
\begin{itemize}
\item 算法工程师养成记(附精选面试题)\url{http://www.ijiandao.com/2b/baijia/157038.html}
\end{itemize}

\subsection{Adaboost}
\label{sec-3-2}
\begin{itemize}
\item 理解AdaBoost算法 \url{https://www.itcodemonkey.com/article/8588.html}
\end{itemize}

\subsection{神经网络}
\label{sec-3-3}

\subsubsection{神经网络的loss function与bp算法求导}
\label{sec-3-3-1}
\begin{itemize}
\item 详解机器学习中的熵、条件熵、相对熵和交叉熵 \url{https://www.cnblogs.com/kyrieng/p/8694705.html}
\item softmax的log似然代价函数(公式求导)\url{https://blog.csdn.net/u014313009/article/details/51045303}
\item 卷积神经网络系列之softmax loss对输入的求导推导 \url{https://blog.csdn.net/weixin_36411839/article/details/81835571}
\end{itemize}

\subsubsection{神经网络梯度下降的方法}
\label{sec-3-3-2}

\subsubsection{Batch Normalization}
\label{sec-3-3-3}
\begin{itemize}
\item 深入理解Batch Normalization批标准化 \url{https://www.cnblogs.com/guoyaohua/p/8724433.html}
\end{itemize}

\subsection{SVM资料}
\label{sec-3-4}
\begin{itemize}
\item SVM支持向量机-拉格朗日乘子与对偶问题(1)\url{https://blog.csdn.net/bit_666/article/details/79865225}
\end{itemize}

\subsection{统计资料}
\label{sec-3-5}
\begin{itemize}
\item 统计学基础--假设检验 \url{https://blog.csdn.net/andy_shenzl/article/details/81453509}
\item 统计学入门(7/7)——假设检验 \url{https://blog.csdn.net/Yi_jia_yi/article/details/81139679}
\end{itemize}

\section{计算机视觉}
\label{sec-4}

\subsection{卷积神经网络}
\label{sec-4-1}
\begin{itemize}
\item 一文概览Inception家族的奋斗史 \url{http://baijiahao.baidu.com/s?id=1601882944953788623&wfr=spider&for=pc}
\item 卷积神经网络的感受野 \url{https://zhuanlan.zhihu.com/p/44106492}
\item 细粒度分类:双线性CNN \url{https://zhuanlan.zhihu.com/p/62532887}
\item 双线性池化(Bilinear Pooling)详解、改进及应用 \url{https://zhuanlan.zhihu.com/p/62532887}
\end{itemize}

\subsection{视觉Attention模型}
\label{sec-4-2}
\begin{itemize}
\item\relax [计算机视觉]深入理解Attention机制 \url{https://blog.csdn.net/yideqianfenzhiyi/article/details/79422857}
\end{itemize}

\subsection{目标检测与目标跟踪}
\label{sec-4-3}
\begin{itemize}
\item 一文读懂Faster RCNN \url{https://zhuanlan.zhihu.com/p/31426458}
\item 目标检测中的Anchor \url{https://zhuanlan.zhihu.com/p/55824651}
\item 目标跟踪简介 \url{https://www.cnblogs.com/jjwu/p/8512730.html}
\item CVPR2018 目标检测(object detection)算法总览 \url{https://blog.csdn.net/niuxinzan/article/details/82107125}
\item 非极大值抑制(合并抛弃重合选区间) \url{https://www.cnblogs.com/makefile/p/nms.html}
\item Anchor Boxes——目标检测质量的关键 \url{https://baijiahao.baidu.com/s?id=1620728785393443256&wfr=spider&for=pc}
\item 如何应用MTCNN和FaceNet模型实现人脸检测及识别 \url{https://cloud.tencent.com/developer/news/213743}
\item CVPR2019目标检测方法进展综述 \url{https://cloud.tencent.com/developer/article/1404879}
\item 目标检测(一)——目标检测综述(持续更新中)\url{https://blog.csdn.net/qq_35451572/article/details/80249259}
\end{itemize}

\section{自然语言处理}
\label{sec-5}

\subsection{Language Model解析}
\label{sec-5-1}
\begin{itemize}
\item 漫谈 Language Model (1): 原理篇 \url{http://blog.pluskid.org/?p=352}
\item 深入理解语言模型 Language Model \url{https://zhuanlan.zhihu.com/p/52061158}
\item Bag-of-words模型入门 \url{https://zhuanlan.zhihu.com/p/29933242}
\item 理解 Word2Vec 之 Skip-Gram 模型 \url{https://zhuanlan.zhihu.com/p/27234078}
\end{itemize}

\subsection{LDA主题模型}
\label{sec-5-2}
\begin{itemize}
\item 自然语言处理基础与实战(8)- 主题模型LDA理解与应用 \url{https://www.jianshu.com/p/74ec7d5f6821}
\end{itemize}

\subsection{NLP Attention模型}
\label{sec-5-3}
\begin{itemize}
\item 模型汇总24 - 深度学习中Attention Mechanism详细介绍:原理、分类及应用 \url{https://zhuanlan.zhihu.com/p/31547842}
\item 深度解析注意力模型(attention model) --- image$_{\text{caption的应用}}$ \url{https://segmentfault.com/a/1190000011744246}
\item 深度学习中的注意力模型(2017版)\url{https://zhuanlan.zhihu.com/p/37601161}
\item 干货|如何在语言翻译中理解Attention Mechanism? \url{http://www.sohu.com/a/156018349_642762}
\item 计算机视觉与NLP结合解释Attention Model \url{https://zhuanlan.zhihu.com/p/56501461}
\item 深入浅出,详细理解Attention Model的基本原理! \url{https://www.jianshu.com/p/ff968920ec73}
\end{itemize}

\subsection{基于Attention的Transformer Model}
\label{sec-5-4}
\begin{itemize}
\item Transformer结构及其应用详解--GPT、BERT、MT-DNN、GPT-2 \url{https://zhuanlan.zhihu.com/p/69290203}
\item 彻底搞懂BERT \url{https://www.cnblogs.com/rucwxb/p/10277217.html}
\item 5 分钟入门 Google 最强NLP模型:BERT \url{https://www.jianshu.com/p/d110d0c13063}
\end{itemize}

\section{学习流程}
\label{sec-6}

\subsection{文本类题}
\label{sec-6-1}
\begin{itemize}
\item 一面
\begin{enumerate}
\item 项目
\item 关键字怎么提取的,TF-IDF有改进么,怎么改进的
\item 命名实体怎么得到的,原理了解
\item LDA的原理是什么,使用了哪个框架
\item 狄利克雷分布能具体说说么
\item 深度学习了解么
\item RNN LSTM了解么
\item 有什么比较熟悉的算法
\item xgboost的原理
\item 有10个排好序的数据库,那么我要找整个的中位数,怎么找
\item 一个路口,一个小时通过一个车的概率是0.9,那么20分钟内通过车的概率是多少
\item 我有一个32位的id是唯一的,那么我想压缩一下,让他还唯一,怎么压缩
\end{enumerate}

\item 二面
\begin{enumerate}
\item 项目
\item SVM原始问题为什么要转化为对偶问题,为什么对偶问题就好求解,原始问题不能求解么
\item K-means 中我想聚成100类 结果发现只能聚成98类,为什么
\item 进程中的内存分段是怎样的
\item 每个线程有哪些东西是自己独享的
\item 一枚不均匀的硬币,我抛了100次,有70次朝上,那么第101次朝上的概率是多少, 这个概率怎么样,公示是如何推导出来的
\item 给你个字符串,字符串是个数字,怎么转换为int型,不用库函数的话
\item 4个海盗,100个金币,每个人轮流提方案,如果你的方案有半数以上通过,那么久可以,否则就会被杀掉,如果你是第一个人,那么你怎么提方案比较好
\item 你的优点是什么
\end{enumerate}
\end{itemize}

\subsection{leetcode 问题题目}
\label{sec-6-2}
\begin{enumerate}
\item 编程重点试一下
\begin{itemize}
\item 95: Unique Binary Search Trees II
\item 96: Unique Binary Search Trees
\end{itemize}
\end{enumerate}

\subsection{机器学习基础}
\label{sec-6-3}

\subsubsection{SIFT特征与HOG特征的表达式与特性}
\label{sec-6-3-1}

\subsubsection{最小二乘推导}
\label{sec-6-3-2}

\subsubsection{逻辑回归推导}
\label{sec-6-3-3}

[./algo-pic/logistic$_{\text{regression}}$.png]

\subsubsection{二层神经网络bp算法推导}
\label{sec-6-3-4}

[./algo-pic/bp.png]

\subsubsection{循环神经网络bp公式推导}
\label{sec-6-3-5}

\subsubsection{BN层公式推导}
\label{sec-6-3-6}

\subsubsection{LSTM公式推导}
\label{sec-6-3-7}

\subsubsection{Softmax bp算法推导}
\label{sec-6-3-8}

[./algo-pic/softmax.png]

\subsubsection{SVM算法推导}
\label{sec-6-3-9}

\subsubsection{AdaBoost算法推导}
\label{sec-6-3-10}

[./algo-pic/adaboost.png]

\subsubsection{GBDT算法推导与特点}
\label{sec-6-3-11}

\subsubsection{谱聚类的推导}
\label{sec-6-3-12}

\subsubsection{PCA与LDA的推导}
\label{sec-6-3-13}

\subsection{计算机视觉}
\label{sec-6-4}
\begin{itemize}
\item BAT机器学习面试1000题系列(第1\textasciitilde{}305题)\url{https://blog.csdn.net/v_JULY_v/article/details/78121924}
\item 深度学习(计算机视觉)面试中问题(一) \url{https://blog.csdn.net/comway_Li/article/details/82532573}
\item 计算机视觉面试常见问题(含解答)\url{https://blog.csdn.net/bluesliuf/article/details/89389117}
\end{itemize}

\subsection{自然语言处理}
\label{sec-6-5}
\begin{itemize}
\item NLP面试(写的比较全面) \url{https://blog.csdn.net/zongza/article/details/82952553}
\item NLP面试 \url{https://blog.csdn.net/qq_24831889/article/details/85876981}
\item NLP面试 \url{https://blog.csdn.net/zdz0200/article/details/82628732}
\item NLP面试 \url{https://blog.csdn.net/qq_28935065/article/details/79075048}
\end{itemize}

\subsection{大数据方向:}
\label{sec-6-6}

\subsubsection{Flink方向技巧:}
\label{sec-6-6-1}
\begin{itemize}
\item waltermark的定义
\item flatMap与map的区别是什么
\item kafka如何保证数据的有序,如果有数据乱序与延迟怎么办
\item 如何保证kafka高性能的消费
\end{itemize}

\subsubsection{Spark方向技巧:}
\label{sec-6-6-2}
\begin{itemize}
\item 宽依赖,窄依赖
\item shuffle的原理
\item hdfs的原理
\end{itemize}

\subsubsection{数据库方向:}
\label{sec-6-6-3}
\begin{itemize}
\item 为什么业务里要使用es,es的优势在哪里:es查询比较灵活,并且可以有聚合操作
\end{itemize}
% Emacs 25.2.2 (Org mode 8.2.10)
\end{document}
